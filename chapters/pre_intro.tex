\section{Pre-session preparation}
\subsection{Installing Python}
	Complete the instructions in this first section before attending the first session.
	
	We assume you are either using a Windows or Mac PC. If you are running Linux then you can probably work it out for yourself! A basic understanding of the terminal (mac) or command prompt (windows) is assumed. If not then Google a tutorial. We assume that you have not previously installed a Python 3 distribution to your system.
	
	Download the Anaconda Python 3.6 distribution from \url{https://www.anaconda.com/download/}. This includes the Python language as well as useful libraries and tools. Install following the instructions and accepting default values.

	Open a terminal window (mac) or command prompt (Windows). Navigate (using the "cd" command to navigate the file system) to an empty directory (I suggest somewhere in My Documents) where you want to store files generated in these lectures. Type:
	\begin{verbatim}
		jupyter notebook
	\end{verbatim}

After a few moments this should open a window in your internet browser. In the top right hand corner click "new" and under notebooks select Python 3. This should open a Jupyter (formerly known as Ipython) notebook. This is an interactive Python session where your code is input into "cells" on the page. Once you have written some code in a cell, press the "Shift" and "Return" keys simultaneously to execute the code in the cell.

To test your distribution has installed correctly type the following into a cell and execute it:

		\begin{lstlisting}[language=Python]
import numpy as np
print("Hello World")\end{lstlisting}
		\begin{verbatim}> Hello World\end{verbatim}
		
In this tutorial, the numbered code on the light grey background is code that you should execute. Bold text is the expected output from that code. Text in \texttt{typewriter font} refers to a Python function. You should attempt all basic tasks in the text. If you find the tasks too simple or have previous experience try the optional advanced tasks for a greater challenge.

\textbf{Ensure that this works correctly before attending the first session.} If in doubt, collaborate with one of your colleagues also taking this course to get it working.