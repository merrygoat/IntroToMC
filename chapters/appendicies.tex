\appendix
\section{Stochastic determination of pi hints}\label{app:pi_hints}
	\subsection{Hint 1}
		Try writing out an algorithm in pseudocode. This means writing out a series of logical steps in simple words first without worrying about the exact programming syntax. For a simple loop this might look something like:

		\begin{lstlisting}
import needed_libraries

total = 0

for loop from 1 to 10
	add loop value to my total
print the total to the screen\end{lstlisting}

		Once you have got the basic program flow sorted, you can then convert your pseudocode to valid Python.

	\subsection{Hint 2}
		This is a basic pseudocode example of the algorithm.

		\begin{lstlisting}
import needed_libraries

num_inside=0
num_interations = 1000
x_coord = 0
y_coord = 0

for loop from 0 to numiterations
	set x_coord using random number
	set y_coord using random number
	check distance of random coord from center
	if distance <= 1 then
		increment num_inside by 1

my_pi = 4* ratio of inside to outside
print my_pi\end{lstlisting}